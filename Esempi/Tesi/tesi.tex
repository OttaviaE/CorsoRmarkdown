% Options for packages loaded elsewhere
\PassOptionsToPackage{unicode}{hyperref}
\PassOptionsToPackage{hyphens}{url}
%
\documentclass[
]{article}
\usepackage{amsmath,amssymb}
\usepackage{lmodern}
\usepackage{iftex}
\ifPDFTeX
  \usepackage[T1]{fontenc}
  \usepackage[utf8]{inputenc}
  \usepackage{textcomp} % provide euro and other symbols
\else % if luatex or xetex
  \usepackage{unicode-math}
  \defaultfontfeatures{Scale=MatchLowercase}
  \defaultfontfeatures[\rmfamily]{Ligatures=TeX,Scale=1}
\fi
% Use upquote if available, for straight quotes in verbatim environments
\IfFileExists{upquote.sty}{\usepackage{upquote}}{}
\IfFileExists{microtype.sty}{% use microtype if available
  \usepackage[]{microtype}
  \UseMicrotypeSet[protrusion]{basicmath} % disable protrusion for tt fonts
}{}
\makeatletter
\@ifundefined{KOMAClassName}{% if non-KOMA class
  \IfFileExists{parskip.sty}{%
    \usepackage{parskip}
  }{% else
    \setlength{\parindent}{0pt}
    \setlength{\parskip}{6pt plus 2pt minus 1pt}}
}{% if KOMA class
  \KOMAoptions{parskip=half}}
\makeatother
\usepackage{xcolor}
\IfFileExists{xurl.sty}{\usepackage{xurl}}{} % add URL line breaks if available
\IfFileExists{bookmark.sty}{\usepackage{bookmark}}{\usepackage{hyperref}}
\hypersetup{
  hidelinks,
  pdfcreator={LaTeX via pandoc}}
\urlstyle{same} % disable monospaced font for URLs
\usepackage[margin=1in]{geometry}
\usepackage{graphicx}
\makeatletter
\def\maxwidth{\ifdim\Gin@nat@width>\linewidth\linewidth\else\Gin@nat@width\fi}
\def\maxheight{\ifdim\Gin@nat@height>\textheight\textheight\else\Gin@nat@height\fi}
\makeatother
% Scale images if necessary, so that they will not overflow the page
% margins by default, and it is still possible to overwrite the defaults
% using explicit options in \includegraphics[width, height, ...]{}
\setkeys{Gin}{width=\maxwidth,height=\maxheight,keepaspectratio}
% Set default figure placement to htbp
\makeatletter
\def\fps@figure{htbp}
\makeatother
\setlength{\emergencystretch}{3em} % prevent overfull lines
\providecommand{\tightlist}{%
  \setlength{\itemsep}{0pt}\setlength{\parskip}{0pt}}
\setcounter{secnumdepth}{5}
\newlength{\cslhangindent}
\setlength{\cslhangindent}{1.5em}
\newlength{\csllabelwidth}
\setlength{\csllabelwidth}{3em}
\newlength{\cslentryspacingunit} % times entry-spacing
\setlength{\cslentryspacingunit}{\parskip}
\newenvironment{CSLReferences}[2] % #1 hanging-ident, #2 entry spacing
 {% don't indent paragraphs
  \setlength{\parindent}{0pt}
  % turn on hanging indent if param 1 is 1
  \ifodd #1
  \let\oldpar\par
  \def\par{\hangindent=\cslhangindent\oldpar}
  \fi
  % set entry spacing
  \setlength{\parskip}{#2\cslentryspacingunit}
 }%
 {}
\usepackage{calc}
\newcommand{\CSLBlock}[1]{#1\hfill\break}
\newcommand{\CSLLeftMargin}[1]{\parbox[t]{\csllabelwidth}{#1}}
\newcommand{\CSLRightInline}[1]{\parbox[t]{\linewidth - \csllabelwidth}{#1}\break}
\newcommand{\CSLIndent}[1]{\hspace{\cslhangindent}#1}
\usepackage{setspace}
\usepackage{multicol}
\usepackage{caption}
\usepackage[italian]{babel}
\captionsetup{format=plain, font=small, labelfont=bf}
\usepackage{graphicx}
\usepackage{subcaption}
\ifLuaTeX
  \usepackage{selnolig}  % disable illegal ligatures
\fi

\author{}
\date{\vspace{-2.5em}}

\begin{document}

\pagenumbering{gobble}

%\begin{titlepage}
	\begin{center}
		\includegraphics[width=0.25\linewidth]{unipd.png}
	\end{center}
	
	\begin{center}
		\begin{Large}
			\textbf{University of Padova}
			
			Department of Philosophy, Sociology, Education, and Applied Psychology (FISPPA)
		\end{Large}
		
	\end{center}
	
	\vspace{3mm}
	\begin{center}
		\begin{large}
			Ph.D. Course in Psychological Sciences (XXXIII Cycle)
		\end{large}
		
		\begin{huge}
			\bfseries
			Inglorious Measures: \\ A Linear Mixed-Effects Model approach for a Rasch analysis of implicit measure accuracy and time responses
		\end{huge}
		
		
	\end{center}
	
	\vspace{2cm}
	\begin{multicols}{2}
		\begin{flushleft}
%			\begin{large}
				\textbf{Advisor:} Prof. Egidio Robusto
				
				\textbf{Co-Advisor:} Prof. Gianmarco Altoè
	%		\end{large}
			
		\end{flushleft}
		\columnbreak
		\begin{flushright}
			\vspace{1.5cm}
			
				\textbf{Ph.D. Candidate:} Ottavia M. Epifania 
			
		\end{flushright}
		
	\end{multicols}

\vspace{2cm}
	
	\begin{center}
		Academic Year: 2019/2020
	\end{center}
	
	

\hypersetup{linkcolor = black}
\newpage
\renewcommand{\contentsname}{Indice}
\pagenumbering{roman}
\tableofcontents
\addcontentsline{toc}{section}{\contentsname}

\newpage

% list of figures have to be added manually to table of contents
\listoffigures % Per toogliere la lista delle figura, aggiungere % all'inizio della riga

\newpage
\listoftables % Per toogliere la lista delle tabelle, aggiungere % all'inizio della riga

\doublespacing

\newpage
\pagenumbering{arabic}
\hypersetup{linkcolor = blue}

\doublespacing

\hypertarget{capitolo-1-introduzione}{%
\section{Capitolo 1: Introduzione}\label{capitolo-1-introduzione}}

\hypertarget{rassegna-della-letteratura}{%
\subsection{Rassegna della
letteratura}\label{rassegna-della-letteratura}}

Come dice Epifania, Anselmi, and Robusto (2021)\ldots..

Se vogliamo evitare di usare \texttt{bookdown} per le cross-reference
(consigliato), possiamo scrivere:

\begin{figure}
\centering
\caption{Il logo unipd.}
\label{fig:logo}

\begin{center}\includegraphics[width=0.5\linewidth]{unipd} \end{center}
\end{figure}

e dire che in Figura \ref{fig:logo} c'è il logo unipd

\hypertarget{scopo-dello-studio}{%
\subsection{Scopo dello studio}\label{scopo-dello-studio}}

\hypertarget{capitolo-2-metodo}{%
\section{Capitolo 2: Metodo}\label{capitolo-2-metodo}}

\hypertarget{capitolo-3-risultati}{%
\section{Capitolo 3: Risultati}\label{capitolo-3-risultati}}

I risultati sono riportati in Tabella \ref{tab:tabella}.

\begin{table}[ht]
\centering
\caption{Tabella dei risultati} 
\label{tab:tabella}
\scalebox{1}{
\begin{tabular}{rll}
  \hline
 &     speed &      dist \\ 
  \hline
X & Min.   : 4.0   & Min.   :  2.00   \\ 
  X.1 & 1st Qu.:12.0   & 1st Qu.: 26.00   \\ 
  X.2 & Median :15.0   & Median : 36.00   \\ 
  X.3 & Mean   :15.4   & Mean   : 42.98   \\ 
  X.4 & 3rd Qu.:19.0   & 3rd Qu.: 56.00   \\ 
  X.5 & Max.   :25.0   & Max.   :120.00   \\ 
   \hline
\end{tabular}
}
\end{table}

In Tabella \ref{tab:modelli} sono riportati i risultati della model
comparison

\begin{table}[!htbp] \centering 
  \caption{Model comparison} 
  \label{tab:modelli} 
\begin{tabular}{@{\extracolsep{5pt}}lcc} 
\\[-1.8ex]\hline 
\hline \\[-1.8ex] 
\\[-1.8ex] & \multicolumn{2}{c}{Distanza} \\ 
\\[-1.8ex] & (1) & (2)\\ 
\hline \\[-1.8ex] 
 Intercetta & 42.98$^{***}$ & $-$17.58$^{**}$ \\ 
  & (3.64) & (6.76) \\ 
  & & \\ 
 Velocità &  & 3.93$^{***}$ \\ 
  &  & (0.42) \\ 
  & & \\ 
\hline \\[-1.8ex] 
Observations & 50 & 50 \\ 
R$^{2}$ & 0.00 & 0.65 \\ 
Adjusted R$^{2}$ & 0.00 & 0.64 \\ 
Residual Std. Error & 25.77 (df = 49) & 15.38 (df = 48) \\ 
F Statistic &  & 89.57$^{***}$ (df = 1; 48) \\ 
\hline 
\hline \\[-1.8ex] 
\textit{Note:}  & \multicolumn{2}{r}{$^{*}$p$<$0.1; $^{**}$p$<$0.05; $^{***}$p$<$0.01} \\ 
\end{tabular} 
\end{table}

\hypertarget{le-figure-doppie}{%
\subsection{Le figure doppie}\label{le-figure-doppie}}

In figura \ref{fig:doppia} si trova un esempio di sottofigure. In Figura
\ref{sub:unipd1} c'è il logo unipd, in Figura \ref{sub:grafico} c'è un
grafico.

\begin{figure}
\centering
\begin{subfigure}{0.3\textwidth}

\begin{center}\includegraphics[width=0.5\linewidth]{unipd} \end{center}
\caption{Di nuovo unipd.}
\label{sub:unipd1}
\end{subfigure}
\begin{subfigure}{0.3\textwidth}

\begin{center}\includegraphics[width=0.8\linewidth]{tesi_files/figure-latex/unnamed-chunk-5-1} \end{center}
\caption{Un grafico.}
\label{sub:grafico}
\end{subfigure}
\caption{Una figura doppia}
\label{fig:doppia}
\end{figure}

\newpage

\hypertarget{bibliografia}{%
\section*{Bibliografia}\label{bibliografia}}
\addcontentsline{toc}{section}{Bibliografia}

\hypertarget{refs}{}
\begin{CSLReferences}{1}{0}
\leavevmode\vadjust pre{\hypertarget{ref-epifania2021implicit}{}}%
Epifania, Ottavia M, Pasquale Anselmi, and Egidio Robusto. 2021.
{``Implicit Social Cognition Through the Years: The Implicit Association
Test at Age 21.''} \emph{Psychology of Consciousness: Theory, Research,
and Practice}.

\end{CSLReferences}

\end{document}
